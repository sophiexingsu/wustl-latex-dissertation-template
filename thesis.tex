\documentclass{wustlthesis}

\setmainfont{Times New Roman}               % Use Times New Roman
\setmathfont{STIXTwoMath}[math-style=ISO]   % Use math font that matches Times New Roman

% Re-calculate the lengths of a text line using the current font
\setlxvchars[\normalfont\normalsize]
\setxlvchars[\normalfont\normalsize]
\checkandfixthelayout[classic]

% Enable refinements of typographics
\usepackage[
    activate=true,
    disable=false,  % enable microtype even in the draft mode
    babel=true,     % enable language-specific kerning.
]{microtype}
\DeclareMicrotypeAlias{Times New Roman}{ptm}

% Hyperlinks
\usepackage{hyperref}
\hypersetup{%
    colorlinks=true,
    linkcolor=black,
    urlcolor=blue,
    citecolor=black,
}
% default URL style
\urlstyle{same}

% Tables (optional packages)
\usepackage{threeparttable}
\usepackage{makecell}  % allow multiple lines in a table cell
\usepackage{multirow}   % allow multiple lines in a table cell

% Graphics
\usepackage{graphicx}

% Subcaptions
% The caption format is "Figure 1. Some caption. (A) subcaption. (B) subcaption".
% And the label reference format is "Figure 1A".
\usepackage{subcaption}
\renewcommand\thesubfigure{\Alph{subfigure}}
\renewcommand\thesubtable{\Alph{subtable}}
\captionsetup{subrefformat=parens}
% Enable subcaption for supplemental figures/tables.
% The caption format is "Supplemental Figure S1. Some caption. (A) …. (B) …."
% and the label reference format is "Figure S1A".
\DeclareCaptionSubType{suppfigure}
\DeclareCaptionSubType{supptable}
\renewcommand\thesubsuppfigure{\Alph{subsuppfigure}}
\renewcommand\thesubsupptable{\Alph{subsupptable}}

% Allowing subcaptions when all figure panels are combined
% into one source image. Based on https://tex.stackexchange.com/a/255790
\newcommand{\phantomlabel}[1]{%
    \parbox{0pt}{\phantomsubcaption\label{#1}}%
}

% Add a note for figure caption spanning multiple pages
\newcommand{\legendcontdnote}{%
    \sourceatright[2em]{%
        \footnotesize\itshape(legend continued on next page)%
}}
\newcommand{\legendcontdref}[1]{%
    \emph{(\fref{#1} continued)}%
}



% Bibliography
% use BibLaTeX + Biber and Nature citation style.
% some extra configurations:
%   - Hide ISBN and URL
%   - Display DOI
%   - Show up to 9 authors
%   - Enable back references
\usepackage[
    backend=biber,
    style=nature,
    date=year,
    isbn=false, url=false, doi=true,
    minnames=1, maxnames=9,
    backref=true
]{biblatex}
% rename bibliography section name
\DefineBibliographyStrings{english}{
    bibliography = {References},
    backrefpage = {cited on p\adddot},
    backrefpages = {cited on pp\adddot}}
% hide PubMed ID (pmid:xxx) in the bibliography
\DeclareFieldFormat{eprint:pmid}{}

% define the bibliography path
\addbibresource{references.bib}



% Configure title page
\settitle{A Mock Thesis on the Proper Formatting of Dissertations and Theses for\\ Arts \& Sciences Graduate Students}
\setauthor{Paige Turner}
\setthesistype{Dissertation}
\setthesisdegree{Doctor of Philosophy}
\setthesisdegreeabbrv{Ph.D.}
% Degree officially earn date must be in December, May, or August
\setdegreedatemy{August}{2022}
\setthesiscommittee{%
    Katherine Davidsen, Chair\\
    Michael Randolf, Co-Chair\\
    Richard Lewis\\
    Hillary O'Connell\\
    Jack Taylor}
\setthesisschool{Division of Biology and Biomedical Sciences}
\setthesisdepartment{Neurosciences}
\setthesispresentee{Washington University in St.\@ Louis}
\setthesisabstractfulldegree{%
    \thesisdegree\ in Biology and Biomedical Sciences\\
    \thesisdepartment}
\setthesisadvisorwithtitle{%
    Professor Katherine Davidsen, Chair\\
    Professor Michael Randolf, Co-Chair}


% Configure PDF metadata
\hypersetup{
    pdftitle={%
        A Mock Thesis on the Proper Formatting of%
        Dissertations and Theses for Arts \& Sciences Graduate Students%
    },
    pdfauthor={\thesisauthor},
}


% Optional packages that are not required in the real thesis
\usepackage[shortlabels]{enumitem}   % Custom list structures (lettered list)
\setlist{noitemsep}
\usepackage{layout}     % Show page layout
\usepackage{pifont}     % Extra symbols (optional if the main font has the symbols)
\usepackage{emoji}      % Emojis

% Optional styling that is not required in the real thesis
\newcommand*{\file}[1]{\texttt{#1}}


\begin{document}

\thetitlepage                       % Title page
\frontmatter
\thesiscopyright{%                  % Thesis copyright page
    \textcopyright~\degreeearnyear, Paige Turner}

\SingleSpacing*
\setSingleSpace{1.15}
\tableofcontents*                   % Table of contents (ToC)
\listoffigures                      % List of figures (LoF)
\listoftables                       % List of tables (LoT)

\DoubleSpacing
\setPagenoteSpacing{1}  % single spacing for footnotes
\thesisacknowledgments

please check out if this is a good thing to be doing right now and if that are good thigns be multiplelied and be happy about the situations, but soemthing is more importnat thnan waht is the things to be worried about. 

\noindent It is appropriate to acknowledge sources of academic and financial support; some fellowships and grants require acknowledgment.

\noindent We offer special thanks to the Washington University School of Engineering for allowing us to use their dissertation template as a starting point for the development of this document.

\null\hfill \thesisauthor

\noindent
\textit{Washington University in St.\@ Louis}\\
\textit{August 2022}
    % Acknowledgements
\thesisdedication{%                 % Dedication page
    Dedicated to my parents.}

\begin{abstract}
After removing these comments, begin typing the body of your abstract here, double-spaced.
Your font should be 12-point (which is the text of this sample paragraph).
No part of the abstract should be bolded.
In the abstract heading above, make sure you use the year your degree is to be officially earned.
Be sure to use your full name as it is recorded in WebSTAC, your dissertation or
thesis advisor's full name(s) wherever appropriate, and the correct title of your degree whenever referencing it.
The title of your degree will not always be the same as the title of your department or program, so please check with your departmental administrative assistant and advisor(s) to be sure you are using the correct degree title.
Please note that an abstract is required for all dissertation submissions in ProQuest.
\end{abstract}

\mainmatter
\pagestyle{main}
\include{thesis-chapter-structure}
\include{thesis-chapter-format}
\chapter{{\LaTeX} test}
\begin{equation}
    \label{eq:maxwell}
    \begin{aligned}
    \frac{\partial\mathcal{D}}{\partial t} & = \nabla\times\mathcal{H},   & \text{(Loi de Faraday)}\\
    \frac{\partial\mathcal{B}}{\partial t} & = -\nabla\times\mathcal{E},  & \text{(Loi d'Ampère)}\\
    \nabla\cdot\mathcal{B}                 & = 0,                         & \text{(Loi de Gauss)}\\
    \nabla\cdot\mathcal{D}                 & = 0.                         & \text{(Loi de Colomb)}
    \end{aligned}
\end{equation}

\[
    \oint_C {E \cdot d\ell = - \frac{d}{{dt}}} \int_S {B_n dA}
\]

L45 = \the\xlvchars\par
L65 = \the\lxvchars

\newpage
\layout


\begin{SingleSpace}
\printbibliography
\end{SingleSpace}

\appendix
\chapter{Degree Program Specification}
\label{app:degree-program}
\textbf{What to Call your Degree Program on your Title Page and your Abstract Page.}

\section{Title page}
The second line (or second and third lines) on the page must name your \emph{administrative unit}.

\begin{itemize}
\zerotrivseps   % eliminate the vertical space introduced by center environment
\item If your degree is offered by one department of Arts \& Sciences on the Danforth Campus, your unit is that department:
    \begin{center}
        \emph{Department of East Asian Languages \& Cultures}
    \end{center}

\item For a co-sponsored degree such as English \& Comparative Literature, credit both:
    \begin{center}
        \emph{Department of English}\\
        \emph{Program in Comparative Literature}
    \end{center}

\item For the Division of Biology \& Biomedical Sciences, credit DBBS and your program:
    \begin{center}
        \emph{Division of Biology \& Biomedical Sciences}\\
        \emph{Neurosciences}
    \end{center}

\item Credit only the program for any of the non-DBBS PhDs on the Medical Campus:
    \begin{center}
        \emph{Interdisciplinary Program in Movement Science}\\
        or\\
        \emph{Program in Speech \& Hearing Sciences}
    \end{center}

\pagebreak
\item If you are in a department in Engineering, credit the School and the department:
    \begin{center}
        \emph{School of Engineering \& Applied Science}\\
        \emph{Department of Biomedical Engineering}
    \end{center}

\item If you are in social work or business, your administrative unit is the School:
    \begin{center}
        \emph{Brown School of Social Work}\\
        or\\
        \emph{Olin Business School}
    \end{center}

\restoretrivseps
\end{itemize}

\section{Abstract page}
Frequent confusion occurs because your abstract heading names your degree rather than your administrative unit, so it may – or may not – match your title page in that respect.

\begin{itemize}
\zerotrivseps   % eliminate the vertical space introduced by center environment

\item For the Division of Biology \& Biomedical Sciences:
    \begin{center}
        \emph{Doctor of Philosophy in Biology and Biomedical Sciences}\\
        \emph{Neurosciences}
    \end{center}

\item For a co-sponsored degree such as English \& Comparative Literature, credit both:
    \begin{center}
        \emph{Doctor of Philosophy in English and Comparative Literature}
    \end{center}

\item If you are in a department in Engineering, credit the School and the department:
    \begin{center}
        \emph{School of Engineering \& Applied Science}\\
        \emph{Department of Biomedical Engineering}
    \end{center}

\item If your degree is offered by one department of Arts \& Sciences on the Danforth Campus:
    \begin{center}
        \emph{Doctor of Philosophy in Chemistry}
    \end{center}

\restoretrivseps
\end{itemize}


\end{document}
